\documentclass[]{report}
\renewcommand{\thesection}{\arabic{section}}
\usepackage{graphicx}
% Title Page
\title{Implementing neural networks with tensorflow report}
\author{Linus Edelkott, Tobias Petri}


\begin{document}
\maketitle

\begin{abstract}
The aim of this project is to create a virtual Texas Hold'em player by using a convolutional neural network, based on the paper "Poker-CNN: a pattern learning strategy for making draws and bets in poker games" \cite{1} by Yakovenko, Nikolai et. al.
The player should finally be able to imitate human poker-playing behaviour, including bluffing, as close as possible - a task where current, non neural network based poker bots usually struggle.
\end{abstract}


\section{Introduction}
Unlike other games such as chess or checkers, poker provides an especially challenging task for computer players (bots). Poker contains a very "human" component aside from its set of rules: the bidding, raising and bluffing. Where other games only provide a reward upon successfully winning them, the reward itself is a key element of the turn structure in poker. Furthermore, Texas Hold'em poker is played with up to 9 people at a regular game, so the state space grows to an enormous size. 
All these factors lead to the fact that current bot implementations are usually rather weak. Confronted with a human player, these bots have the huge disadvantage of being predictable: They often show repetitive behaviour and fail to bluff successfully. Self learning systems though can put an end to this, as the Libratus project shows\cite{2}: Developed at the Carnegie Mellon University, Libratus was able to beat 4 pro poker players in a 20 day tournament. After initially losing, the bot changed its strategy after day 4. Analogous to Libratus, the team of Yakovenko, Nikolai et. al. developed a CNN based approach that aims at surpassing the current implementations and is even competitive with human tournament players \cite{1}. Their bost is able do to so, using only simple bots as their data source for generating a large training and validation set. Because the network is only learning from successful games, the CNN is afterwards able to easily surpass the weak bots.    

\section{Poker-CNN: A Pattern Learning Strategy for Making Draws and Bets in Poker Games}
\subsection{The original paper}

In the paper ``A Pattern Learning Strategy for Making Draws and Bets
in Poker Games''\cite{1} the authors propose a convolutional neural
network, which should be adaptable for all poker variants under the
hypothesize that poker games can be described as a pattern matching
problem. The poker network learns through iterative self-play and
improves using the results of its previous actions for training. The
main challenges for a data driven approach like this one, is to find
a good representation of the game and to arrive at a sophisticated result using self-generated,
imperfect data.

In order to solve the first challenge, the authors introduce a unified
representation of poker games, which should be adaptable for several poker variants. To encode the game state into a form
which can be used in a convolutional neural network (CNN) the game information
is described in different matrices. There are 13 ranks of poker\footnote{2,3,4, ..., J,Q,K,A}
cards and 4 different suits\footnote{club, diamond, heart, spade},
so each card is represented in a 4 x 13 sparse binary matrix, where
only one element is non zero. To capture the full hand information
a further layer is added, the sum of the previous layers. The advantages
in doing so are: a large input builds a good base for a CNN. Additionally the
full hand representations makes it easier to model common poker patterns, even
``without game specific card sorting or suit isomorphisms\footnote{e.g. AsKd
is essentially the same as KhAc}''.\cite{1} Because poker is not
only expressed in cards, context information need to be passed to
the CNN as well. For example the number of chips in the pot is represented
as a numerical coded 4x13 matrix, where the minmal amount is coded
similar to the smallest poker card and the maximal amount\footnote{4{*}13{*}minimal amount (small blind), everything higher than this
is encoded identical} is encoded as a binary matrix (entails only ones). A binary 4x13
matrix (for two-player games) is used to represent the position, hence
it entails whether the system is first to bet. Further layers are
added, which are not relevant for Texas Hold'em. Each matrix is finally
zero padded to a 17x17 matrix to ease the convolutions and max pooling.
The whole Yx17x17 tensor is used as input. In section \ref{see:adapted_p}
we discuss our extended representation for Texas Hold'em no Limit
Cash Game.

Different poker games usually consist of either betting or draw actions,
in rare cases there are both type of actions. 

Draw actions are defined as actions in which the player has the option
to replace one or more of his cards with a new drawn card. The task
for a neural network concerning draws is to estimate the return of
every possible card replacement and select the one with the highest
output. Yakovenko, Nikolai et. al. employed Monte Carlo Sampling to
first generate 250000 poker hands and simulate the expected outcome
for each possible draw, in each of these hands.

While the 250000 hands are encoded into the input tensor, the expected
outcome will serve as label. The authors did train three different
networks: a fully connected neural network, a convolutional layer
with a filter size of 5x5 and a convolutional neural network with
the filter size of 3x3. The latter one outperformed the other two,
hence we used this one as basis. Further details of the network will
be discussed in section \ref{see:CNN_arc}. 

Betting actions on the other hand are defined as action in which the
player evaluates the win chance of his current hand and places a corresponding
amount of chips. If he sees no winning chance at all, he can quit the
round whereby he loses his already placed bets. Important to
notice is that players don't have to be honest about their hands and may
raise (bet a higher amount of chips than the previous players) regardless
in order to bluff the other players into folding. Due to this fact
there is often not ``one'' right move. Hence players often vary their
actions although their cards might be similar. 

In the case of Texas Hold'em limited Poker, which was used in the
paper as trainings example, the player has 5 betting actions:

Check - bet nothing (but keep playing), which is only possible if
their are no placed bets in this round yet

Bet - start to actually put something in the pot

Raise - set a higher amount of chips than the previous players

Call - match the bet of the previous players

Fold - give up the current hand

Due to this different options the expected outcome can not simply
be estimated by Monte Carlo sampling. Thus for each training hand
several epochs of this particular hand were simulated using simple heuristic
bots, which adjusted their winning chance according to the previously
simulated Allin probabilities. Once again the expected outcome of
this simulation was used as label for the respective hand. Overall
500000 hands were used for training. 

As a result the CNN was finally able to outperform the bot used for
training and was even competetive against a human expert.

\subsection{Input adaption for (regular) Texas Hold'em \label{see:adapted_p} }

Texas Hold'em Poker is one of the most popular poker variants existing. We wanted to examine how the presented CNN would perform
trained on actual Texas Hold'em states. In fact Yakovenko, Nikolai et. al. already implemented
Texas Hold'em states as training data set, but they used a simplified variant instaed of the most common one.

Our initial goal was to implement the CNN on training data containing
the game states of (regular) ``9 player Texas Hold'em no Limit (cash-game)''.
The main difference hereby, aside from the player count, is that in
limited poker the player may only bet and raise a fixed amount.

In general Texas Hold'em consists of 4 betting rounds. At first each
player gets 2 cards, known as hole cards, and the first betting round
begins. Afterwards 3 community cards (the flop) are drawn which are
visible for everyone and constitute the current hand of each player
together with their respective hole cards. The next betting round
starts and the game continues with the drawing of a single community
card and the following betting round which will be repeated once more
(the turn and river). The 5 best cards out of this 7 constitute
the hand of each player and determines the winner.

Instead of using a 4x13 matrix for each individual card we used one
matrix for each round. Thus the first 4x13 matrix has two elements
different than zero, the second one has 3 and the others have one,
respectively. Hence the network shouldn't differentiate between the
sequence of cards entailed in one round (e.g. 5s5c is identically
to 5c5s). The layer containing all cards information (the sum of the
previous 7) will be kept. Furthermore using one binary matrix as input
is no longer sufficient. A 9x9 matrix will be used instead, where
each row despite one is zero. The row filled with ones marks
the position (who is first to bet). Another 9x9 matrix entails information
about the state of the other players, if the respective row is 1 they
are still active and didn't fold yet. Additional to the numerical
coded amount of chips in the pot a second similarly coded matrix containing
the amount of chips placed in the current betting round is needed.
We assume that a really competitive Player would need more information
but this would probably also increase the learning phase. The final
representation is a 9x17x17 tensor, summarized in Table \ref{tab:tensor}.

\begin{table}
\includegraphics[scale=0.5]{Features.JPG}

\caption{Features used as inputs for Texas Hold'em no Limit Poker \label{tab:tensor}}

\end{table}
\subsection{The network structure \label{see:CNN_arc} }

The convolutional neural network that was used within this project has an architecture similar to the original CNN from the paper\cite{1}. While Yakovenko, Nikolai et. al. compared a fully connected network, a CNN with 3x3 filters and a 5x5 filter CNN, We decided to only compare 3x3 and 5x5 CNNs. Tensorflow gave us the advantage of using 3d convolution, unlike the 2d convolution used in the original network\footnote{The paper does not specify the exact type of convolution used. However, based on the github projects of the author, we assume 2d convolution.}. With the input as explained in section 2.2, the first two layers are 3d convolutional, with a max pool layer after the second convolution. This is followed by yet another 2 convolutional layers with max pooling at the end, only with increased size. 
Afterwards, a dense 50\% dropout layer feeds directly into the output layer. Figure 1 shows the network structure from the paper. Aside from the different input, the output of our implementation has also been modified: Where the original had 5 output neurons, our version has only 2, representing the options 'fold' and 'call'. These are the only relevant moves, since our training opponent is an allin bot.

The cost function for this network is the mean squared error, with the Nesterov momentum as optimization function. Both were chosen according to the original paper and compared to an Adam optimizer as well as cross entropy, showing that the MSE together with Nesterov perform better.


\begin{figure}[h]
	\caption{Network structure}
	\includegraphics[scale = 0.5]{cnn_structure.jpg}
\end{figure} 

We made a few critical changes to the original paper:
The original paper created the playing hand first and created several
simulations of this particular hand as label. All of this happened
in the training process. Our approach however generated one hand and
simulated the results of each possible move beforehand and only once.
So while the original network used an evaluation of the expected outcome,
our network should learn to predict the outcome of this move based
on one single simulation (assuming that the hand was only generated
once), increasing the complexity of the task immensely. We realised
this mistake relatively late and given the way our poker engine was implemented,
changing the simulation results from one to several simulations would
have resulted in a calculation time unfit for our computer hardware. We assume
that this adaptation of the simulations would already be sufficient
to finally win against a simple Allin player.

The second change was that we decreased the input tensor, Yakovenko,
Nikolai et. al. used mostly full binary matrices, while we encoded
the position and active players with 9x9 matrices where one row is
always zero or 1. It might be more fitting to used 18 4x13 (padded
to 17x17) matrices, because ``a large input creates a good capacity
for building convolutional layers''\cite{1}. Admittedly this change
would increase the calculation time even more.

\section{Evaluation \label{see:evaluation} }

Unfortunately we were not able to produce any significant results and our network was unable to learn how to beat even a simple Allin bot. In order to find the source of error we tried several different settings. However it didn't make a difference if the used kernel were 5, 16 or 32, neither did it make a difference whether we used the whole tensor as input, or only the hole cards. We assume that this is caused by varying reasons. 

First of all, even 65000 hands are probably not enough to train the if we compare it to the 2598960 possible hole cards in Poker. However, even when run on the same hands repeatedly (several epochs) the network did not succeed in classifying a winning move. Assuming that the amount of data was not the problem, another reason might have been the data format. This assumption could be reasonable for the first implementation with 10 labels, but is rather unlikely for the final version with only 2 outputs. Since the input that was used in the paper differs from ours, this seems to be the most likely cause right now. As far as the actual network structure goes, different versions with varying layer size, amount of layers, cost functions and optimizers all yielded a very similar result, only decreasing in accuracy. It should be noted that all networks showed no significant increase in accuracy, but rather a random behaviour within a normal distribution. This leads to the conclusion that the network completely played by chance without any real strategy and it didn't learn anything significant from the training data. This is rather unexpected, given the results of the paper, but can be explained by our changes.
Concluding, a similar network with improved input data, the numbers of samples as well the format, together with including more simulations to evaluate the expected outcome  seems like the most promising solution.

\begin{thebibliography}{}
	\bibitem{1} Yakovenko, Nikolai, et al., \emph{Poker-CNN: a pattern learning strategy for making draws and bets in poker games.}, arXiv preprint arXiv:1509.06731 (2015).
	
	\bibitem{2} Rivercasino.com, \emph{Brain VS AI},
	riverscasino.com/pittsburgh/BrainsVsAI

\end{thebibliography}  



\end{document}  
